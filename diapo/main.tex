% Permission is granted to copy, distribute and/or modify this
% document under the terms of the Creative Common by-nc-sa License
% version 3.0 (CC BY-NC-SA 3.0). A copy of the license can be found at
% http://creativecommons.org/licenses/by-nc-sa/3.0/legalcode.
%

%% Slides Class
\documentclass[10pt,a4paper]{beamer}

%% Setting beamer style
\usepackage{UBdx}

\setbeamertemplate{bibliography entry title}{}
\setbeamertemplate{bibliography entry location}{}
\setbeamertemplate{bibliography entry note}{}

%% Font packages
\usepackage{lmodern}
\usepackage[T1]{fontenc}
\usepackage[utf8]{inputenc}
\usepackage{babel}
\usepackage[colorinlistoftodos]{todonotes}
\usepackage{listings}
\usepackage{amssymb}
\usepackage{amsmath}
\usepackage{amsthm}
\usepackage{appendix}
\usepackage{multicol}
\usepackage{stmaryrd}
\usepackage{verbatim}
\usepackage{algorithm, algorithmic}

\theoremstyle{plain}
\newtheorem{thm}{Théorème}[section]
\newtheorem{lemme}[thm]{Lemme}
\newtheorem{propo}[thm]{Proposition}

\theoremstyle{definition}
\newtheorem{remarque}[thm]{Remarque}
\newtheorem{defi}[thm]{Définition}
\newtheorem{nota}[thm]{Notation}


\renewcommand{\a}{\alpha}
\newcommand{\HRule}{\rule{\linewidth}{0.5mm}} 
\newcommand{\F}{\mathbb{F}} 
\newcommand{\A}{\mathcal{A}} 
\newcommand{\e}{\mathbf{e}}  
\newcommand{\s}{\mathbf{s}} 
\newcommand{\K}{\mathbb{K}} 
\newcommand{\J}{\mathcal{J}} 
\newcommand{\D}{\mathcal{D}} 

% traduction des mots pour les algos
\floatname{algorithm}{Algorithme}
\renewcommand{\algorithmicrequire}{\textbf{Entrées:}}
\renewcommand{\algorithmicensure}{\textbf{Sortie:}}
\renewcommand{\algorithmicrepeat}{\textbf{Faire}}
\renewcommand{\algorithmicuntil}{\textbf{Tant que}}
\renewcommand{\algorithmicreturn}{\textbf{Retourne}}

% Highlight macros
\newcommand{\highlight}[1]{\textcolor{structure.fg}{\bfseries #1}}

%% Title, subtitle, authors, institute, date, ...
\title{Projet - Wave}
\subtitle{Un procédé de signature à base de codes correcteurs}
\author{ Suzanne LANSADE, Eva PALANDJIAN \\\quad\quad\quad\quad\quad \\Encadrant : Gilles ZEMOR}


\institute[Master CSI]{Master CSI, Université de Bordeaux, France}

\date{Mardi 25 Février 2020}

%%%%%%%%%%%%%%%%%%%%%%%%%%[ Document ]%%%%%%%%%%%%%%%%%%%%%%%%%%
\begin{document}

\begin{frame}
  \vspace{3.5em}
  \titlepage

\end{frame}

\begin{frame}
  \frametitle{Plan}
  \tableofcontents[subsectionstyle=hide]
\end{frame}

%%%%%%%%%%%%%%%%%%%%%%%%%%%%%%%%%%%%%%%%%%%%%%%%%%%%%%%%%%%%%%%%%%%
\section{Introduction}
\begin{frame}
\frametitle{Introduction}
\begin{itemize}
\item[•] arrivée ordinateur quantique -> alternative RSA + DH 
\item[•] code correcteur -> quantiquement sûr 
\item[•] chiffrement ok signature pas ok
\item[•] NIST
\end{itemize}
\end{frame}

\begin{frame}
\frametitle{Introduction}
\begin{figure}[h]
\begin{center}
\includegraphics [scale=0.3]{../rapport/include/nist_second_tour.png}
\end{center}
\caption{\small Comparaison des soumissions au NIST du second tour.}
\end{figure}
\end{frame}

\begin{frame}
\frametitle{Introduction}
Wave solution au problème pour les signatures
poids grands
\end{frame}

%%%%%%%%%%%%%%%%%%%%%%%%%%%%%%%%%%%%%%%%%%%%%%%%%%%%%%%%%%%%%%%%%%%%%%
\section{Schéma de signature Wave}
\begin{frame}<handout:0>
  \frametitle{Plan}
  \tableofcontents[currentsection,subsectionstyle=hide]
\end{frame}

\subsection{Codes $(U,U+V)$-généralisés}

\begin{frame}
\frametitle{Codes $(U,U+V)$-généralisés}
\begin{defi} 
%Soient $n$ un entier pair et $a$,$b$,$c$,$d$ quatre vecteurs de $\F_q^{n/2}$ tels que $\forall i \in \llbracket 1,n/2\rrbracket$ :
%\begin{equation}
%a_ic_i \neq 0 
%\end{equation}
%\begin{equation}
%a_id_i - b_ic_i \neq 0 
%\end{equation}

\noindent Soient $n$ un entier pair, $U$ et $V$ deux codes définis comme précédemment. Le code $(U,U+V)$-généralisé correspond à l'ensemble :
\begin{center}
$\{(a.u + b.v, c.u + d.v)$ tel que $u \in U$ et $v \in V \}$
\end{center}
où $x.y$ est le produit coordonnée par coordonnée des $x_i$ et $y_i$ et $a$,$b$,$c$,$d$ sont quatre vecteurs de $\F_q^{n/2}$.\\
\end{defi}
\end{frame}


\subsection{Signature}

\begin{frame}
\frametitle{Signature}
\begin{itemize}
\item fonction syndrome : sens unique
\item trappe : structure du code
\end{itemize}
\end{frame}

\begin{frame}[fragile]
\frametitle{Signature}
\begin{multicols}{2}
\begin{flushleft}
$\verb|Sign|^{sk}(s)$:\\
	$\quad$ e $\leftarrow  \verb|InvertAlg|(s,T)$ \\
	$\quad \verb|renvoie| e$\\
\end{flushleft}
\begin{flushleft}
$\verb|Verify|^{pk}(s,e')$: \\
	$\quad \verb| Si | e'H^T = s \verb| et | |e'| = \omega $ \\
	$\quad \quad \verb|renvoie 1| $\\
	$\quad \verb|renvoie 0|$\\
\end{flushleft}
\end{multicols}
\end{frame}

\section{Décodage}

\begin{frame}
\frametitle{Décodage}
décodage classique\\
décoder UV <=> décoder U puis décoder V\\

\end{frame}

\begin{frame}
\frametitle{Décodage}
- poids haut -> conditions sur les coordonnées de $e_U$ avant le décodage\\
- alors e aura un grand poids
\end{frame}


\begin{frame}
\frametitle{Décodage}
\begin{figure}[h]
\begin{center}
\includegraphics [scale=0.32]{../rapport/include/graph_ratio_w.png}
\end{center}
\caption{\scriptsize Comparaison des distances $w/n$ avec et sans trappe en fonction du rendement.}
\end{figure}
\end{frame}

%%%%%%%%%%%%%%%%%%%%%%%%%%%%%%%%%%%%%%%%%%%%%%%%%%%%%%%%%%%%%%%%%%%%%%
\section{Uniformisation des sorties}

\begin{frame}
\frametitle{Uniformisation des sorties}
fuite d'info\\
m1 donne les info sur la permutation\\
\end{frame}


\begin{frame}
\frametitle{La méthode du rejet}
\begin{figure}[h]
\begin{center}
\includegraphics [scale=0.32]{algo_UV.png}
\end{center}
\end{figure}
\end{frame}

\begin{frame}
\frametitle{La méthode du rejet}
preuve générique rejet
\end{frame}

\begin{frame}
\frametitle{La méthode du rejet}
application à nous
\end{frame}

%%%%%%%%%%%%%%%%%%%%%%%%%%%%%%%%%%%%%%%%%%%%%%%%%%%%%%%%%%%%%%%%%%%%%%
\section{Sécurité EUF-CMA}

\begin{frame}[fragile]
\frametitle{Sécurité EUF-CMA}	
\begin{figure}[h]
		\includegraphics[scale=0.35]{eufcma.png}
\end{figure}
\end{frame}


\begin{frame}[fragile]
\noindent Le jeu EUF-CMA se déroule comme suit. $\mathcal{A}$ fait appel à \verb|Init|. Il peut ensuite faire $N_{sign}$ requêtes à \verb|sign|. Le jeu est dit réussi si $\mathcal{A}$ est capable de donner $(s,e)$ accepté par \verb|Fin| et tel que $s$ n'est jamais été demandé à \verb|Sign|. \\
On définit alors le succès EUF-CMA comme :
$$Succ^{EUF-CMA}_{Wave}(t,N_{sign}) := \max_{\mathcal{A};|A|\leq t}(\mathbb{P}(\mathcal{A}\text{ réussit le jeu EUF-CMA de Wave})).$$
Le protocole est alors sûr au sens EUF-CMA si ce succès est négligeable. \\
\end{frame}

\begin{frame}
\frametitle{Sécurité EUF-CMA}
\textbf{Le problème DOOM.} Soient des paramètres $(n,q,k,\omega,N)$, où $N$ est un entier. \\

\leftskip=1cm
\noindent \textbf{I :} $\mathbf{H}$ une matrice uniforme de $\F_q^{(n-k)\times n}$ et $(\mathbf{s}_1,...,\mathbf{s}_N)$ une liste de $N$ syndromes. 

\noindent \textbf{Q :} Décoder l'un des syndromes à la distance $w := \lfloor \omega n \rfloor$. \\

\leftskip=0cm

\noindent On définit alors le succès de DOOM comme :
$$Succ^{DOOM(n,q,k,N)}(t) := \max_{\mathcal{A};|A|\leq t}(\mathbb{P}(\mathcal{A}(\mathbf{H},\mathbf{s}_1,...,\mathbf{s}_n)=\mathbf{e}\text{ tel que }$$
$$ \mathbf{eH}^T = \mathbf{s}_j \text{ pour un certain } j \in \{1,...,N\})).$$

%pourquoi à DOOM et pas au decodage?\\
%expliquer pourquoi ça parait logique\\
\end{frame}

%\begin{frame}
%\frametitle{Sécurité EUF-CMA}
%expliquer réduction : jeux à donner
%\end{frame}


%%%%%%%%%%%%%%%%%%%%%%%%%%%%%%%%%%%%%%%%%%%%%%%%%%%%%%%%%%%%%%%%%%%%%%
\section{Implémentation et résultats}

\begin{frame}
\frametitle{Implémentation et résultats}
\frametitle{Résultats de notre implémentation}
\begin{center}
\begin{tabular}{|c|c|c|c|} 
\hline
\textbf{nombre d'itérations} & \textbf{d} & \textbf{nombre de rejets} & \textbf{ratio\;} \\\hline 
100 & 0 & 6 & 6\% \\\hline
100 & 1 & 3 & 3\% \\\hline 
100 & 2 & 6 & 6\% \\\hline
100 & 3 & 3 & 3\% \\\hline
100 & 4 & 6 & 6\% \\\hline
100 & 5 & 4 & 4\% \\\hline
\end{tabular}
\end{center}
\begin{center}
\begin{tabular}{|c|c|c|c|} 
\hline
\textbf{nombre d'itérations} & \textbf{d} & \textbf{nombres de rejets} & \textbf{ratio} \\\hline 
400 & 3 & 19 & $\sim$5\% \\\hline
400 & 5 & 17 & $\sim$4\%\\\hline 
\end{tabular}\

\vspace{0.2in}
\textbf{Ratio moyen de l'article : $\sim$10\%}
\end{center}
\end{frame}


%%%%%%%%%%%%%%%%%%%%%%%%%%%%%%%%%%%%%%%%%%%%%%%%%%%%%%%%%%%%%%%%%%%%%%
\section{Références}

%\begin{frame}<handout:0>
%  \frametitle{Plan}
%  \tableofcontents[currentsection,subsectionstyle=hide]
%\end{frame}



\begin{frame}
  \frametitle{Références}
  \begin{enumerate}
  \item[{[1]}] Thomas Debris-Alazard. \textit{Cryptographie fondée sur les codes : nouvelles approches pour constructions et preuves ; contribution en cryptanalyse}. 2019.
  \vspace{0.1in}
  \item[{[2]}] Jean-Pierre Tillich, Thomas Debris-Alazard, Nicolas Sendrier. \textit{Wave : A
new family of trapdoor one-way preimage sampleable functions based on codes}. 2018.
  \end{enumerate}
\end{frame}


\end{document}