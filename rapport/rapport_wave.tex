\documentclass[12pt]{article}
\usepackage{babel}
\usepackage[utf8x]{inputenc}
\usepackage[T1]{fontenc}
\usepackage[colorinlistoftodos]{todonotes}
\usepackage{listings}
\usepackage{amssymb}
\usepackage{amsmath}
\usepackage{amsthm}
\usepackage{appendix}

\theoremstyle{definition}
\newtheorem{thm}{Théorème}[section]
\newtheorem{lemme}[thm]{Lemme}
\newtheorem{remarque}[thm]{Remarque}
\newtheorem{defi}[thm]{Définition}
\newtheorem{propo}[thm]{Proposition}

\renewcommand{\a}{\alpha}
\newcommand{\HRule}{\rule{\linewidth}{0.5mm}} 
\newcommand{\F}{\mathbb{F}} 
\newcommand{\K}{\mathbb{K}} 

\title{Projet - Wave}
\author{Lansade Suzanne}
\author{Palandjian Eva}

\begin{document}

\begin{titlepage}

\center

\textsc{\LARGE Université de Bordeaux}\\[2.0cm]
\textsc{\Large Master 2 : Cryptologie et Sécurité Informatique}\\[0.5cm] 
\textsc{\large Projet de fin d'études}\\[1.2cm] 

\HRule \\[0.4cm]
{ \huge \bfseries Wave - Un procédé de signature \\ à base de codes correcteurs}\\[0.3cm] 
\HRule \\[1.3cm]


\begin{minipage}{0.4\textwidth}
\begin{flushleft} \large
Suzanne \textsc{Lansade}\\
Eva \textsc{Palandjian}\\
\end{flushleft}
\end{minipage}
~
\begin{minipage}{0.4\textwidth}
\begin{flushright} \large
\emph{Encadrant:} \\
Gilles \textsc{Zemor}
\end{flushright}
\end{minipage}\\[2cm]

\begin{large}
Février, 2020
\end{large}

\vspace{0.5in}

\includegraphics [scale=0.2]{Universite_Bordeaux.jpg}

\end{titlepage}

\newpage
\tableofcontents
\newpage

\section*{Introduction}
\addcontentsline{toc}{section}{\protect\numberline{}Introduction}

- pb post-quantique
- appel d'offre NIST
	-> tableau : aucun code correcteur en signatures
- dur de trouver l'ensemble des syndromes facilement décodable
- dur de créer une fonction de hachage qui envoie m dans l'ensemble des syndromes possibles
-> pb décodage NP-complet
- mot y de syndrome s est associé à un unique mot de code c le plus proche de y
quand on chiffre -> ok
quand on signe -> pas ok car dur de trouver un syndrome de cet sorte
du coup la solution de machinx est d'enlever la restriction au mot le plus proche
-> Wave innovation 
Nous allons détailler le schéma de signature Wave et détailler sa sécurité.

\section{Le schéma de signature Wave}
\subsection{La famille de code (U,U+V) généralisé}
\subsection{Le principe de signature}
fonctions gpvm : syndrome
\subsection{Le décodage avec trappe}
\subsection{Implémentation et choix de paramètres}

\section{Fuite d'information -> rejet}
\subsection{Une fuite d'information}
\subsection{La méthode du rejet}
preuve de e uniformément distribué
\subsection{Estimation du nombre de rejet}
\subsection{Une famille de fonctions uniformément distribuée}

\section{Sécurité du schéma}
\subsection{Sécurité EUF-CMA}
réduction au pb DOOM
-> à la fin ajouter que même avec la fonction de hachage, la réduction reste possible
\subsection{Distinction d'une matrice de parité d'un code (U,U+V) et d'une matrice aléatoire}
réduction à un pb NP-complet
ne pas oublier de rappeler que normalement on ajoute S et P pour masquer la forme de la matrice.

\section*{Conclusion}
\addcontentsline{toc}{section}{\protect\numberline{}Conclusion}

\newpage


\bibliographystyle{plain}
\bibliography{bibliography}

\end{document}