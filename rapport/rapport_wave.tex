\documentclass[12pt]{article}
\usepackage{babel}
\usepackage[utf8x]{inputenc}
\usepackage[T1]{fontenc}
\usepackage[colorinlistoftodos]{todonotes}
\usepackage{listings}
\usepackage{amssymb}
\usepackage{amsmath}
\usepackage{amsthm}
\usepackage{appendix}

\theoremstyle{definition}
\newtheorem{thm}{Théorème}[section]
\newtheorem{lemme}[thm]{Lemme}
\newtheorem{remarque}[thm]{Remarque}
\newtheorem{defi}[thm]{Définition}
\newtheorem{propo}[thm]{Proposition}

\renewcommand{\a}{\alpha}
\newcommand{\HRule}{\rule{\linewidth}{0.5mm}} 
\newcommand{\F}{\mathbb{F}} 
\newcommand{\K}{\mathbb{K}} 

\title{Projet - Wave}
\author{Lansade Suzanne}
\author{Palandjian Eva}

\begin{document}

\begin{titlepage}

\center

\textsc{\LARGE Université de Bordeaux}\\[2.0cm]
\textsc{\Large Master 2 : Cryptologie et Sécurité Informatique}\\[0.5cm] 
\textsc{\large Projet de fin d'études}\\[1.2cm] 

\HRule \\[0.4cm]
{ \huge \bfseries Wave - Un procédé de signature \\ à base de codes correcteurs}\\[0.3cm] 
\HRule \\[1.3cm]


\begin{minipage}{0.4\textwidth}
\begin{flushleft} \large
Suzanne \textsc{Lansade}\\
Eva \textsc{Palandjian}\\
\end{flushleft}
\end{minipage}
~
\begin{minipage}{0.4\textwidth}
\begin{flushright} \large
\emph{Encadrant:} \\
Gilles \textsc{Zemor}
\end{flushright}
\end{minipage}\\[2cm]

\begin{large}
Février, 2020
\end{large}

\vspace{0.5in}

\includegraphics [scale=0.2]{Universite_Bordeaux.jpg}

\end{titlepage}

\newpage
\tableofcontents
\newpage

\section*{Introduction}
\addcontentsline{toc}{section}{\protect\numberline{}Introduction}

- passage au post-quantique \\
- appel d'offre NIST \\
	-> tableau : aucun code correcteur en signatures \\
- dur de trouver l'ensemble des syndromes facilement décodable \\
- dur de créer une fonction de hachage qui envoie m dans l'ensemble des syndromes possibles \\
- problème du décodage NP-complet \\
- mot y de syndrome s est associé à un unique mot de code c le plus proche de y \\
quand on chiffre -> ne pose pas de problème \\
quand on signe -> pose un problème car il est dur de trouver un syndrome de cette sorte \\
la solution Wave est d'enlever la restriction au mot le plus proche \\
Nous allons détailler le schéma de signature Wave et détailler sa sécurité. \\

\section{Le schéma de signature Wave}
Pour répondre aux problèmes : \\
- Définition des codes (U,U+V)-généralisés \\
- Des fonctions GPV en moyenne \\
- Un schéma de signature de type hash et signe utilisant ces codes \\

\subsection{La famille de codes (U,U+V)-généralisés}
Définition des codes (U,U+V)-généralisés: \\
- Comment les créer  \\
- Choix des paramètres a,b,c,d \\
- Liens entre les matrices des codes U et V et du code UV \\
- Les dimensions et différents paramètres \\
- Calcul du hull ==> q > 2 \\
- ...? \\

\subsection{Le principe de signature}
Un schéma hash et signe utilisant la fonction syndrome comme fonction à sens unique : \\
- Définition des fonctions GPVM, un couple (Trapdoor, InvertAlg) où trapdoor est un algo poly proba renvoyant une matrice de parité et la trappe associée, et où InvertAlg est un algo poly proba prenant en entrée la trappe et renvoyant l'inverse de la fonction syndrome. \\
De plus, ces fonctions sont (1) bien distribuées, (2) sans fuite d'info en moyenne, (3) sens unique sans la trappe \\
- Le schéma : un algo signe et un algo verify.

\subsection{Le décodage avec trappe}
Détail de l'algorithme invertAlg avec utilisation de la trappe: \\
- Conditions sur le poids de e: \\
--> facile \\
--> facile avec trappe \\
--> difficile \\
- Inverser le syndrome sur le code UV <==> inverser le syndrome sur U et sur V et prendre son image par Phi. \\
- Prendre un ev par un algo de décodage quelconque, utiliser les propriétés du code UV pour en déduire un eu, vérifier le poids de e, recommencer. \\
- Différences gros poids et petits poids \\

\subsection{Implémentation et choix de paramètres}
TODO \\

\section{Uniformisation des signatures et syndromes}

\subsection{Une fuite d'information}
- Malheureusement, fuite d'information en raison des correspondance entre e[i] et e[i+n/2] ! \\
- Calcul de proba \\
- Pourquoi c'est problématique \\ 

\subsection{La méthode du rejet}
Idée générale : On attend pour que les sorties aient l'air uniforme (soient suffisemment proches de l'uniforme): \\
- On choisit ev de façon a ce qu'il soit uniforme dans son ensemble \\
- On met des conditions de rejet sur eu en fonction de ev pour que eu ait l'air uniforme \\
- On obtient un e qui a l'air uniforme \\

\subsection{Estimation du nombre de rejet}
TODO \\

\subsection{Une famille de fonctions uniformément distribuée}
On a donc le point (2) de la definition des fonctions GPV qui est obtenu dans la section précédente. On va montrer le point (1), à savoir, notre famille de fonctions syndrômes est uniformément distribuée avec les codes (U,U+V)-généralisés \\

\section{Sécurité du schéma}

\subsection{Sécurité EUF-CMA}
\subsubsection{Définitions}
\subsubsection{Réduction au problème DOOM}
\subsubsection{Et la fonction de hachage ?}

\subsection{Indistinguabilité des codes (U,U+V)-généralisés}
Distinguer une matrice de parité d'un code (U,U+V)-généralisé d'une matrice de parité aléatoire. \\
Réduction à un problème NP-complet. \\
Utilisation de S et P pour masquer les propriétés de la matrice. \\

\section*{Conclusion}
\addcontentsline{toc}{section}{\protect\numberline{}Conclusion}

\newpage


\bibliographystyle{plain}
\bibliography{bibliography}

\end{document}