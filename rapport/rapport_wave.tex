\documentclass[12pt]{article}
\usepackage{babel}
\usepackage[utf8x]{inputenc}
\usepackage[T1]{fontenc}
\usepackage[colorinlistoftodos]{todonotes}
\usepackage{listings}
\usepackage{amssymb}
\usepackage{amsmath}
\usepackage{amsthm}
\usepackage{appendix}

\theoremstyle{definition}
\newtheorem{thm}{Théorème}[section]
\newtheorem{lemme}[thm]{Lemme}
\newtheorem{remarque}[thm]{Remarque}
\newtheorem{defi}[thm]{Définition}
\newtheorem{propo}[thm]{Proposition}

\renewcommand{\a}{\alpha}
\newcommand{\HRule}{\rule{\linewidth}{0.5mm}} 
\newcommand{\F}{\mathbb{F}} 
\newcommand{\K}{\mathbb{K}} 

\title{Projet - Wave}
\author{Lansade Suzanne}
\author{Palandjian Eva}

\begin{document}

\begin{titlepage}

\center

\textsc{\LARGE Université de Bordeaux}\\[2.0cm]
\textsc{\Large Master 2 : Cryptologie et Sécurité Informatique}\\[0.5cm] 
\textsc{\large Projet de fin d'études}\\[1.2cm] 

\HRule \\[0.4cm]
{ \huge \bfseries Wave - Un procédé de signature \\ à base de codes correcteurs}\\[0.3cm] 
\HRule \\[1.3cm]


\begin{minipage}{0.4\textwidth}
\begin{flushleft} \large
Suzanne \textsc{Lansade}\\
Eva \textsc{Palandjian}\\
\end{flushleft}
\end{minipage}
~
\begin{minipage}{0.4\textwidth}
\begin{flushright} \large
\emph{Encadrant:} \\
Gilles \textsc{Zemor}
\end{flushright}
\end{minipage}\\[2cm]

\begin{large}
Février, 2020
\end{large}

\vspace{0.5in}

\includegraphics [scale=0.2]{Universite_Bordeaux.jpg}

\end{titlepage}

\newpage
\tableofcontents

\newpage
\section*{Introduction}
\addcontentsline{toc}{section}{\protect\numberline{}Introduction}


\bibliographystyle{plain}
\bibliography{bibliography}

\end{document}